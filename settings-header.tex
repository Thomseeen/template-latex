\documentclass[a4paper, 12pt, headsepline]{scrreprt}


%% ***** Preamble *****
%% ---------------------------------------------------------------------------
%% Hauptpakete packages
\usepackage[english, ngerman]{babel} 		% Deutsche typogr. Regeln + Trenntabelle
\usepackage[T1]{fontenc}             		% Interner TeX-Font-Codierung
\usepackage{lmodern}                 		% Font Latin Modern
\usepackage[utf8]{inputenc}          		% Font-Codierung der Eingabedatei
\usepackage[babel, german=quotes]{csquotes}	% Anführungszeichen
\usepackage{graphicx}                		% Grafiken
\usepackage{booktabs}                		% Tabellen schöner
\usepackage{longtable}						% Tabellen über mehrere Seiten
\usepackage{multirow}						% Multirow Tabellen
\usepackage{makecell}						% Um Padding innerhalb einer Zelle zu erlauben
\setcellgapes{10pt}							% - padding um x
\usepackage{listingsutf8}					% Listings mit Einstellungen
\usepackage{blindtext}						% Generierung von Blindtext
\lstset{basicstyle=\small\ttfamily,
	tabsize=2,
	basewidth={0.5em,0.45em},
	extendedchars=true}
\usepackage{amsmath}	               		% Mathematik
\usepackage[pdftex]{hyperref}       
\hypersetup{
	bookmarksopen=true,
	bookmarksopenlevel=3,
	colorlinks,
	citecolor=blue,
	linkcolor=blue,
}
\usepackage{scrhack}						% Unterdrückt Fehlermeldung von listings
\usepackage{subfigure}						% Abbildungen mit Unterabbildungen
\usepackage{caption}						% Anpassbare Captions
\usepackage{indentfirst}					% Nach Überschrift auch Einrücken
%% ---------------------------------------------------------------------------
%% Kopfzeile mit Section und Subsection
%\usepackage{titleps}
%\usepackage{scrlayer-scrpage}
%\lohead{\thesection. \sectiontitle}
%\rohead{\thesubsection. \rightmark}
%% ---------------------------------------------------------------------------
%% Verzeichnisse
\usepackage[nohyperlinks, printonlyused]{acronym}			% Für Abkürzungsverzeichnis
%% ---------------------------------------------------------------------------
%% BibTeX
\usepackage{cite}
\def\BibTeX{Bib}
%% ---------------------------------------------------------------------------
%% Nummerierungstiefen
\setcounter{tocdepth}{3}             		% 3 Stufen im Inhaltsverzeichnis
\setcounter{secnumdepth}{3} 		 		% 3 Stufen in Abschnittnummerierung
%% ---------------------------------------------------------------------------
%% Form
% Disable single lines at the start of a paragraph (Schusterjungen)
\clubpenalty = 10000
% Disable single lines at the end of a paragraph (Hurenkinder)
\widowpenalty = 10000
\displaywidowpenalty = 10000
% Packet für Seitenrandabstände und Einstellung für Seitenränder
%\usepackage{geometry}
%\geometry{left=2.54cm, right=2.54cm, top=2.54cm, bottom=2.54cm} % aus Word
%% ---------------------------------------------------------------------------
%% Quelltext
%\usepackage{color}
%\definecolor{light-gray}{gray}{0.95}
%\usepackage{listings} 
%\lstset{
%	numbers=left, 
%	numberstyle=\tiny, 
%	breaklines=true,
%	backgroundcolor=\color{light-gray},
%	numbersep=5pt
%}
%% ---------------------------------------------------------------------------
%% Sonder Gedöns
% Macro für die "schöne" Darstellung von C#
\newcommand{\Csharp}{
	{\settoheight{\dimen0}{C}C\kern-.05em \resizebox{!}{\dimen0}{\raisebox{\depth}{\#}}}}
%% ---------------------------------------------------------------------------